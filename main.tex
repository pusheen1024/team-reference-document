\documentclass[10pt,twocolumn,oneside]{article}
\usepackage{preamble}

\begin{document}
\pagestyle{fancy}
\fancyfoot{}
\fancyhead[L]{Saratov State University: Druzhkin, Anofrikova, Potapkina}
\fancyhead[R]{\thepage}

\section{Data structures}
\subsection{Treap (with segment updates)}
\inputminted[breaklines]{cpp}{Data structures/treap (segment updates).cpp}
\subsection{Treap (as a binary search tree)}
\inputminted[breaklines]{cpp}{Data structures/treap (as a binsearch tree).cpp}
\subsection{Segment tree}
\inputminted[breaklines]{cpp}{Data structures/segment tree.cpp}
\subsection{Segment tree with segment updates}
\inputminted[breaklines]{cpp}{Data structures/segment tree with segment updates.cpp}
\subsection{Merge sort tree}
\inputminted[breaklines]{cpp}{Data structures/mergesort tree.cpp}
\subsection{Sparse table}
\inputminted[breaklines]{cpp}{Data structures/sparse table.cpp}
\subsection{DSU}
\inputminted[breaklines]{cpp}{Data structures/dsu.cpp}
\subsection{Min queue}
\inputminted[breaklines]{cpp}{Data structures/min queue.cpp}
\subsection{Priority queue}
\inputminted[breaklines]{python}{Data structures/priority queue.py}
\subsection{PBDS}
\inputminted[breaklines]{cpp}{Data structures/pbds.cpp}

\section{String algorithms}
\subsection{Trie}
\inputminted[breaklines]{cpp}{String algorithms/trie.cpp}
\subsection{Suffix array}
\inputminted[breaklines]{cpp}{String algorithms/suffix array.cpp}
\subsection{String hashing}
\inputminted[breaklines]{cpp}{String algorithms/string hashing.cpp}
\subsection{KMP}
\inputminted[breaklines]{python}{String algorithms/kmp.py}
\subsection{Prefix function}
\inputminted[breaklines]{python}{String algorithms/prefix-function.py}
\subsection{Z"=function}
\inputminted[breaklines]{python}{String algorithms/z-function.py}

\section{Graph algorithms}
\subsection{Djikstra}
\inputminted[breaklines]{cpp}{Graph algorithms/djikstra.cpp}
\subsection{Floyd}
\inputminted[breaklines]{cpp}{Graph algorithms/floyd.cpp}
\subsection{Ford"=Bellman}
\inputminted[breaklines]{python}{Graph algorithms/ford-bellman.py}
После n"=ной итерации что"=то изменилось $\Rightarrow$ \\
существует цикл отрицательного веса.
\subsection{Cycle in directed graph}
\inputminted[breaklines]{cpp}{Graph algorithms/cycle in directed graph.cpp}
\subsection{Strongly connected components}
\inputminted[breaklines]{cpp}{Graph algorithms/strongly connected components.cpp}
\subsection{Bridges}
\inputminted[breaklines]{cpp}{Graph algorithms/bridges.cpp}
\subsection{Articulation points}
\inputminted[breaklines]{cpp}{Graph algorithms/articulation points.cpp}
\subsection{Triangle counting}
\inputminted[breaklines]{cpp}{Graph algorithms/triangles.cpp}
\subsection{Binary lifting, LCA, min on path}
\inputminted[breaklines]{cpp}{Graph algorithms/binary lifting + lca + min on path.cpp}
\subsection{LCA with RMQ}
\inputminted[breaklines]{cpp}{Graph algorithms/lca with rmq.cpp}
\subsection{Virtual tree}
\inputminted[breaklines]{cpp}{Graph algorithms/virtual tree.cpp}
\subsection{MST}
\inputminted[breaklines]{cpp}{Graph algorithms/mst.cpp}
\subsection{Rerooting}
\inputminted[breaklines]{cpp}{Graph algorithms/rerooting.cpp}
\subsection{Maximum matching on trees}
\inputminted[breaklines]{cpp}{Graph algorithms/maximum matching on trees.cpp}
\subsection{Maximum matching (Kuhn)}
\inputminted[breaklines]{cpp}{Graph algorithms/kuhn (with optimizations).cpp}
\subsection{Path cover, edge cover (Kuhn)}
\inputminted[breaklines]{cpp}{Graph algorithms/path cover + edge cover (kuhn).cpp}
\subsection{2"=SAT}
\inputminted[breaklines]{cpp}{Graph algorithms/2-sat.cpp}
\subsection{Flows}
\inputminted[breaklines]{cpp}{Graph algorithms/flows.cpp}
\subsection{DCP}
\inputminted[breaklines]{cpp}{Graph algorithms/dcp.cpp}

\section{SQRT heuristics}
\subsection{Sqrt decomposition}
\inputminted[breaklines]{python}{SQRT heuristics/sqrt decomposition.py}
\subsection{Query decomposition}
\inputminted[breaklines]{cpp}{SQRT heuristics/query decomposition.cpp}
\subsection{Mo's algorithm}
\inputminted[breaklines]{cpp}{SQRT heuristics/mo's algorithm.cpp}

\section{DP}
\subsection{LIS}
\inputminted[breaklines]{cpp}{DP/lis.cpp}
\subsection{LCS}
\inputminted[breaklines]{python}{DP/lcs.py}
\subsection{Sums with certificate}
\inputminted[breaklines]{python}{DP/sums with certificate.py}
\subsection{Log trick}
\inputminted[breaklines]{cpp}{DP/log-trick.cpp}
\subsection{Bitset optimization}
\inputminted[breaklines]{cpp}{DP/bitset optimization.cpp}
\subsection{Matrix optimization}
\inputminted[breaklines]{cpp}{DP/matrix optimization.cpp}
\subsection{Meet"=in"=the"=middle}
\inputminted[breaklines]{cpp}{DP/meet-in-the-middle.cpp}
\subsection{SOS DP}
\inputminted[breaklines]{cpp}{DP/sos-dp.cpp}
\subsection{Digit DP}
\inputminted[breaklines]{cpp}{DP/digit dp.cpp}
\subsection{Fence}
\inputminted[breaklines]{cpp}{DP/fence.cpp}

\section{Permutations}
\subsection{Inversions (merge sort)}
\inputminted[breaklines]{cpp}{Permutations/inversions.cpp}
\subsection{K"=th permutation}
\inputminted[breaklines]{cpp}{Permutations/k-th permutation.cpp}
\subsection{Permutation index}
\inputminted[breaklines]{python}{Permutations/permutation number.py}
\subsection{Permutation cycles}
\inputminted[breaklines]{cpp}{Permutations/permutation cycles.cpp}
\subsection{Number of operations to sort a permutation}
\begin{itemize}
\item Разрешено менять местами 2 соседних элемента "--- количество инверсий;
\item Разрешено менять местами 2 любых элемента "--- N минус количество циклов в перестановке;
\item Разрешено вставить любой элемент на другую позицию "--- N минус длина НВП;
\end{itemize}

\section{Maths}
\subsection{Modular arithmetics}
\inputminted[breaklines]{cpp}{Maths/modular arithmetics.cpp}
\subsection{CRT}
\inputminted[breaklines]{python}{Maths/crt.py}
\subsection{The ultimate geometry template}
\inputminted[breaklines]{python}{Maths/geometry.py}
\subsection{FFT}
\inputminted[breaklines]{python}{Maths/fft.py}
\subsection{Eratosphenes sieve}
\inputminted[breaklines]{cpp}{Maths/sieve.cpp}
\subsection{Segmented sieve}
\inputminted[breaklines]{cpp}{Maths/segmented sieve.cpp}
\subsection{Factorization with sieve}
\inputminted[breaklines]{cpp}{Maths/factorization with sieve.cpp}
\subsection{Divisors generation}
\inputminted[breaklines]{cpp}{Maths/divisors.cpp}
\subsection{Mobius function}
\inputminted[breaklines]{cpp}{Maths/mobius.cpp}

\section{Other algorithms}
\subsection{XOR hashing}
\inputminted[breaklines]{cpp}{Other algorithms/xor-hashing.cpp}
\subsection{Coordinate compression}
\inputminted[breaklines]{cpp}{Other algorithms/coordinate compression.cpp}
\subsection{Construction with no more than k equal elements next to each other}
\inputminted[breaklines]{cpp}{Other algorithms/no equal neighbours.cpp}
\subsection{Real"=number binary search}
\inputminted[breaklines]{cpp}{Other algorithms/float binsearch.cpp}
\subsection{Parallel binary search}
\inputminted[breaklines]{cpp}{Other algorithms/parallel binsearch.cpp}
\subsection{Sprague"=Grundy}
\inputminted[breaklines]{cpp}{Other algorithms/sprague-grundy.cpp}

Функция Гранди для комбинации игр = XOR-сумма функций Гранди для каждой из игр.

XOR-сумма = 0 $\Rightarrow$ выигрывает 2ой игрок.

XOR-сумма $\ne 0 \Rightarrow$ выигрывает 1ый игрок.

\section{Important formulas}
\documentclass{article}
\usepackage[T2A]{fontenc}
\usepackage[utf8]{inputenc}
\usepackage{amsthm}
\usepackage{amsmath}
\usepackage{amssymb}
\usepackage{amsfonts}
\usepackage{mathrsfs}
\usepackage[12pt]{extsizes}
\usepackage{fancyvrb}
\usepackage{indentfirst}
\usepackage[
  left=2cm, right=2cm, top=2cm, bottom=2cm, headsep=0.2cm, footskip=0.6cm, bindingoffset=0cm
]{geometry}
\usepackage[english,russian]{babel}


\begin{document}

\section*{Математические формулы}
Линейность матожидания: $M(x_1 + x_2 + \dots + x_n) = M(x_1) + M(x_2) + \dots + M(x_n)$ \\
Сумма геометрической прогрессии: $S_n = \frac{b_1 (1 - q^n)}{1 - q}$ \\
Сумма бесконечно убывающей геометрической прогрессии: $S = \frac{b_1}{1 - q}$ \\
Сумма арифметической прогрессии: $S_n = \frac{a_1 + a_n}{2} \cdot n = \frac{2a_1 + d(n - 1)}{2} \cdot n$ \\
$1^2 + 2^2 + 3^2 + \dots + n^2 = \frac{n(n(n + 1)(2n + 1)}{6}$ \\
$1^3 + 2^3 + 3^3 + \dots + n^3 = \frac{n^2 (n + 1)^2}{4}$ \\
Площадь криволинейного сектора: $\mu(F) = \frac{1}{2} \int_{\alpha}^{\beta} r^2(\varphi) d\varphi$ \\
$ \int \sqrt{a^2 - x^2} dx = \frac{x}{2} \sqrt{a^2 - x^2} + \frac{a^2}{2} arcsin \frac{x}{a} + C $ \\
$ \int \sqrt{x^2 \pm a^2} dx = \frac{x}{2} \sqrt {x^2 \pm a^2} \pm \frac{a^2}{2} ln |x + \sqrt{x^2 \pm a^2}| + C $ \\
$ \int \frac{dx}{a^2 + x^2} = \frac{1}{a} arctg \frac{x}{a} + C (a \ne 0) $ \\
$ \int \frac{dx}{a^2 - x^2} = \frac{1}{2a} ln |\frac{a + x}{a - x}| + C (a \ne 0) $ \\
$ \int \frac{x dx}{a^2 \pm x^2} = \pm \frac{1}{2} ln |a^2 \pm x^2| + C $ \\
$ \int \frac{dx}{\sqrt{a^2 - x^2}} = arcsin \frac{x}{a} + C (a \ne 0) $ \\
$ \int \frac{dx}{\sqrt{x^2 \pm a^2}} = ln |x + \sqrt{x^2 \pm a^2}| + C $ \\
$ \int \frac{x dx}{\sqrt{a^2 \pm x^2}} = \pm \sqrt{a^2 \pm x^2} + C $ \\

\end{document}



\end{document}

