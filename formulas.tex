\documentclass{article}
\usepackage[T2A]{fontenc}
\usepackage[utf8]{inputenc}
\usepackage{amsthm}
\usepackage{amsmath}
\usepackage{amssymb}
\usepackage{amsfonts}
\usepackage{mathrsfs}
\usepackage[12pt]{extsizes}
\usepackage{fancyvrb}
\usepackage{indentfirst}
\usepackage[
  left=2cm, right=2cm, top=2cm, bottom=2cm, headsep=0.2cm, footskip=0.6cm, bindingoffset=0cm
]{geometry}
\usepackage[english,russian]{babel}


\begin{document}

\section*{Математические формулы}
Линейность матожидания: $M(x_1 + x_2 + \dots + x_n) = M(x_1) + M(x_2) + \dots + M(x_n)$ \\
Сумма геометрической прогрессии: $S_n = \frac{b_1 (1 - q^n)}{1 - q}$ \\
Сумма бесконечно убывающей геометрической прогрессии: $S = \frac{b_1}{1 - q}$ \\
Сумма арифметической прогрессии: $S_n = \frac{a_1 + a_n}{2} \cdot n = \frac{2a_1 + d(n - 1)}{2} \cdot n$ \\
$1^2 + 2^2 + 3^2 + \dots + n^2 = \frac{n(n(n + 1)(2n + 1)}{6}$ \\
$1^3 + 2^3 + 3^3 + \dots + n^3 = \frac{n^2 (n + 1)^2}{4}$ \\
Площадь криволинейного сектора: $\mu(F) = \frac{1}{2} \int_{\alpha}^{\beta} r^2(\varphi) d\varphi$ \\

\end{document}

